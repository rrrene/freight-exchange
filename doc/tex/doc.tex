
%Achtung: Wenn eps Grafiken im Dokument enthalten sind von pdftex auf tex und ghostsript umstellen (Im Men� Setzen von TeXShop)
%In dieser Datei sind die Graphiken auskommentiert.

%Hier steht die Pr�ambel
\documentclass[a4paper,parskip=half*,12pt]{scrartcl} 
%definiert Papiergr��e und Dokumentklasse (KOMA Script)

%Die folgenden Zeilen sind der �bersichtlichkeit halber so gestaltet
%Alternativ lassen sich auch mehrere Pakte zusammenfassen:
%\usepackage{jurabib,hyperref,graphicx,setspace}

\usepackage[ngerman]{babel}
\usepackage[utf8]{inputenc}
%Erlaubt die direkte Eingabe von z.B. Umlauten, zur Aktivierung das %-Zeichen entfernen

\usepackage[paper=a4paper,left=40mm,right=10mm,top=25mm,bottom=10mm]{geometry} 

\usepackage[T1]{fontenc}

\usepackage{hyperref}
%Ein paar weitere Pakete f�r dies und das ;-)

\usepackage{graphicx} 
%Wenn Bilder eingef�gt werden sollen muss da drinstehen

\usepackage{setspace}
%Wird f�r den Zeilenabstand ben�tigt

%einzelne, herrenlose Zeilen (Hurenkinder und Schusterjungen) vermeiden
\clubpenalty = 10000
\widowpenalty = 10000
\displaywidowpenalty = 10000

%wie weit sollen paragraphen einger�ckt werden
\setlength{\parindent}{0em}

%Auf die Einr�ckung der Fu�noten wird verzichtet
\deffootnote{1em}{1em}{%
\textsuperscript{\thefootnotemark\ }}

%Bis zu welcher Gliederungsebene soll nummeriert werden 
\setcounter{secnumdepth}{3}

\setcounter{tocdepth}{3}


%Kopf- und Fu�zeile
\usepackage{fancyhdr}
\pagestyle{fancy} 
\fancyhf{} 
\fancyhead[L]{\nouppercase{\leftmark}}%Kopfzeile links, zentriert: [C] 
\fancyhead[R]{\thepage} %Kopfzeile rechts, links:

\usepackage{color}
\newcommand{\bad}[1]{\textcolor{red}{#1}}

% Ak�rzungen und Acronyme kennzeichnen
\newcommand{\aav}[2]{#1}
\newcommand{\quelle}[1]{\textsc{#1}}
%\aav{vgl.}{vergleiche}
%\aav{S.}{Seite}
%\aav{et al.}{et alii}
%\aav{f.}{folgende}
%\aav{ff.}{fortfolgende}
%\aa%v{o.S.}{ohne Seitenangabe}
%\aa%v{o.V.}{ohne Verfasserangabe}
%\aav{}{}
%

%Jetzt beginnt das eigentliche Dokument
\begin{document}

%ab jetzt sollen die Seitenzahlen beginnen, mit 1 f�r die erste Seite, usw.
\setcounter{page}{1}
\pagenumbering{arabic}

%ab jetzt anderthalbfacher Zeilenabstand
\onehalfspacing

% This file was generated Sun Oct 31 11:50:40 +0100 2010
%!TEX root = ../c:/Dokumente und Einstellungen/rf/github/diplom.rails/lib/tasks/../../doc/tex/doc.tex

\begin{table}[h]
\caption{Felder der "`app\_configs"'-Tabelle}
\begin{tabular*}{}{|l|l|p{7cm}|l|} \hline
  \textbf{Name} & \textbf{Typ}  & \textbf{Beschreibung}  & \textbf{Beispiel} \tabularnewline \hline

  name & String & Name der Konfigurationsvariable & language \tabularnewline \hline

  value & Text & Wert der Konfigurationsvariable & de \tabularnewline \hline

\end{tabular*}
\label{tab:schema_app_configs}
\end{table}

% This file was generated Wed Nov 03 17:02:41 +0100 2010
%!TEX root = ../c:/Dokumente und Einstellungen/rf/github/diplom.rails/lib/tasks/../../doc/tex/doc.tex

\begin{table}[h]
  \begin{center}
    \begin{tabular}{|p{.3\textwidth}|p{.09\textwidth}|p{.26\textwidth}|p{.24\textwidth}|} \hline
      \textbf{Name} & \textbf{Typ}  & \textbf{Beschreibung}  & \textbf{Beispiel} \tabularnewline \hline

      name & String & Name der Firma & "`Mustermann GmbH"' \tabularnewline \hline

      phone & String & Festnetznummer & "`0123-5436895"' \tabularnewline \hline

      fax & String & Faxnummer & "`0123-5436896"' \tabularnewline \hline

      mobile & String & Mobilfunknummer & "`0123-5436897"' \tabularnewline \hline

      email & String & E-Mail-Adresse & "`mm@example.org"' \tabularnewline \hline

      website & String & URL der Website & "`www.example.org"' \tabularnewline \hline

      address & String & Addresse & "`"' \tabularnewline \hline

      address2 & String & Addresse (Fortsetzung) & "`"' \tabularnewline \hline

      zip & String & Postleitzahl & "`"' \tabularnewline \hline

      city & String & Ort & "`"' \tabularnewline \hline

      country & String & Land & "`"' \tabularnewline \hline

      misc & Text &  & "`"' \tabularnewline \hline

      contact\_person\_id & Integer & Verweis auf die Person, die Ansprechpartner sein soll. & 1 \tabularnewline \hline

    \end{tabular}
    \caption{Struktur der "`companies"'-Datenbanktabelle}
    \label{tab:schema_companies}
  \end{center}
\end{table}

% This file was generated Sun Oct 31 11:50:40 +0100 2010
%!TEX root = ../c:/Dokumente und Einstellungen/rf/github/diplom.rails/lib/tasks/../../doc/tex/doc.tex

\begin{table}[h]
\caption{Felder der "`countries"'-Tabelle}
\begin{tabular*}{}{|l|l|p{7cm}|l|} \hline
  \textbf{Name} & \textbf{Typ}  & \textbf{Beschreibung}  & \textbf{Beispiel} \tabularnewline \hline

  name & String &  &  \tabularnewline \hline

  iso & String &  &  \tabularnewline \hline

\end{tabular*}
\label{tab:schema_countries}
\end{table}

% This file was generated Wed Nov 03 17:02:41 +0100 2010
%!TEX root = ../c:/Dokumente und Einstellungen/rf/github/diplom.rails/lib/tasks/../../doc/tex/doc.tex

\begin{table}[h]
  \begin{center}
    \begin{tabular}{|p{.3\textwidth}|p{.09\textwidth}|p{.35\textwidth}|p{.15\textwidth}|} \hline
      \textbf{Name} & \textbf{Typ}  & \textbf{Beschreibung}  & \textbf{Beispiel} \tabularnewline \hline

      user\_id & Integer & Verweis auf den Benutzer, dem das Objekt gehört. & 1 \tabularnewline \hline

      company\_id & Integer & Verweis auf die Firma, zu der das Objekt gehört. & 1 \tabularnewline \hline

      origin\_site\_info\_id & Integer & Verweis auf den Startort & 34 \tabularnewline \hline

      destination\_site\_info\_id & Integer & Verweis auf den Zielort & 35 \tabularnewline \hline

      weight & Integer & Gewicht der Fracht (in t) & 50 \tabularnewline \hline

      loading\_meter & Integer & Lademeter & 30 \tabularnewline \hline

      hazmat & Boolean & Ist das Gut ein Gefahrgut? & true \tabularnewline \hline

      transport\_type & String & Art der Wagen & single\_wagon \tabularnewline \hline

      wagons\_provided\_by & String & Wer stellt die Wagen bereit? & railway \tabularnewline \hline

      desired\_proposal\_type & String & Welche Art von Angebot wird gewünscht? & package\_price \tabularnewline \hline

      contact\_person\_id & Integer & Verweis auf die Person, die Ansprechpartner sein soll. & 1 \tabularnewline \hline

    \end{tabular}
    \caption{Struktur der "`freights"'-Datenbanktabelle}
    \label{tab:schema_freights}
  \end{center}
\end{table}

% This file was generated Thu Nov 04 14:11:13 +0100 2010
%!TEX root = ../c:/Dokumente und Einstellungen/rf/github/diplom.rails/lib/tasks/../../doc/tex/doc.tex

\begin{table} %[h]
  \begin{center}
    \begin{tabular}{|p{.3\textwidth}|p{.09\textwidth}|p{.35\textwidth}|p{.15\textwidth}|} \hline
      \textbf{Name} & \textbf{Typ}  & \textbf{Beschreibung}  & \textbf{Beispiel} \tabularnewline \hline

      user\_id & Integer & Verweis auf den Benutzer, dem das Objekt gehört. & 1 \tabularnewline \hline

      company\_id & Integer & Verweis auf die Firma, zu der das Objekt gehört. & 1 \tabularnewline \hline

      origin\_site\_info\_id & Integer & Verweis auf den Startort & 34 \tabularnewline \hline

      destination\_site\_info\_id & Integer & Verweis auf den Zielort & 35 \tabularnewline \hline

      weight & Integer & Gewicht der Fracht (in t) & 50 \tabularnewline \hline

      loading\_meter & Integer & Lademeter & 30 \tabularnewline \hline

      hazmat & Boolean & Ist das Gut ein Gefahrgut? & true \tabularnewline \hline

      transport\_type & String & Art der Wagen & single\_wagon \tabularnewline \hline

      contact\_person\_id & Integer & Verweis auf die Person, die Ansprechpartner sein soll. & 1 \tabularnewline \hline

    \end{tabular}
    \caption{Struktur der "`loading\_spaces"'-Datenbanktabelle}
    \label{tab:schema_loading_spaces}
  \end{center}
\end{table}

% This file was generated Wed Nov 03 21:38:09 +0100 2010
%!TEX root = ..//Users/rene/Sites/github/diplom.rails/lib/tasks/../../doc/tex/doc.tex
\begin{table}[h]
  \begin{center}
    \begin{tabular}{|l|p{.09\textwidth}|p{.4\textwidth}|l|} \hline
      \textbf{Name} & \textbf{Typ}  & \textbf{Beschreibung}  & \textbf{Beispiel} \tabularnewline \hline
      item\_type & String & Verweis auf den Typ des Objekts, zu dem die Information gehört & "`Freight"' \tabularnewline \hline
      item\_id & Integer & Verweis auf die ID des Objekts, zu dem die Information gehört & 1 \tabularnewline \hline
      name & String & Name & "`misc\_text"' \tabularnewline \hline
      lang & String & Sprache & "`en"' \tabularnewline \hline
      text & Text & Text & "`We are proud to..."' \tabularnewline \hline
    \end{tabular}
    \caption{Struktur der "`localized\_infos"'-Datenbanktabelle}
    \label{tab:schema_localized_infos}
  \end{center}
\end{table}

% This file was generated Wed Nov 03 21:38:09 +0100 2010
%!TEX root = ..//Users/rene/Sites/github/diplom.rails/lib/tasks/../../doc/tex/doc.tex
\begin{table}[h]
  \begin{center}
    \begin{tabular}{|l|p{.09\textwidth}|p{.4\textwidth}|l|} \hline
      \textbf{Name} & \textbf{Typ}  & \textbf{Beschreibung}  & \textbf{Beispiel} \tabularnewline \hline
      a\_type & String & Verweist auf den Typ des A-Objekts & "`Freight"' \tabularnewline \hline
      a\_id & Integer & Verweist auf die ID des A-Objekts & 182 \tabularnewline \hline
      b\_type & String & Verweist auf den Typ des B-Objekts & "`LoadingSpace"' \tabularnewline \hline
      b\_id & Integer & Verweist auf die ID des B-Objekts & 98 \tabularnewline \hline
      result & Float & Das Resultat des Vergleichs & 0.765 \tabularnewline \hline
    \end{tabular}
    \caption{Struktur der "`matching\_recordings"'-Datenbanktabelle}
    \label{tab:schema_matching_recordings}
  \end{center}
\end{table}

% This file was generated Thu Nov 04 14:11:13 +0100 2010
%!TEX root = ../c:/Dokumente und Einstellungen/rf/github/diplom.rails/lib/tasks/../../doc/tex/doc.tex

\begin{table} %[h]
  \begin{center}
    \begin{tabular}{|p{.3\textwidth}|p{.09\textwidth}|p{.26\textwidth}|p{.24\textwidth}|} \hline
      \textbf{Name} & \textbf{Typ}  & \textbf{Beschreibung}  & \textbf{Beispiel} \tabularnewline \hline

      first\_name & String & Vorname & "`Max"' \tabularnewline \hline

      last\_name & String & Nachname & "`Mustermann"' \tabularnewline \hline

      gender & String & Geschlecht & "`male"' \tabularnewline \hline

      job\_description & String & Dienstbezeichnung & "`Vertriebsleiter Nord"' \tabularnewline \hline

      phone & String & Festnetznummer & "`0123-5436895"' \tabularnewline \hline

      fax & String & Faxnummer & "`0123-5436896"' \tabularnewline \hline

      mobile & String & Mobilfunknummer & "`0123-5436897"' \tabularnewline \hline

      email & String & E-Mail-Adresse & "`mm@example.org"' \tabularnewline \hline

      website & String & URL der Website & "`www.example.org"' \tabularnewline \hline

      locale & String & Sprache, in der die Person die Benutzeroberfläche nutzt & "`de"' \tabularnewline \hline

    \end{tabular}
    \caption{Struktur der "`people"'-Datenbanktabelle}
    \label{tab:schema_people}
  \end{center}
\end{table}

% This file was generated Sun Oct 31 11:50:40 +0100 2010
%!TEX root = ../c:/Dokumente und Einstellungen/rf/github/diplom.rails/lib/tasks/../../doc/tex/doc.tex

\begin{table}[h]
\caption{Felder der "`recordings"'-Tabelle}
\begin{tabular*}{}{|l|l|p{7cm}|l|} \hline
  \textbf{Name} & \textbf{Typ}  & \textbf{Beschreibung}  & \textbf{Beispiel} \tabularnewline \hline

  item\_type & String &  &  \tabularnewline \hline

  item\_id & Integer &  &  \tabularnewline \hline

  action & String &  &  \tabularnewline \hline

  diff & Text &  &  \tabularnewline \hline

  user\_id & Integer & Verweis auf den Benutzer, dem das Objekt gehört. & 1 \tabularnewline \hline

  company\_id & Integer & Verweis auf die Firma, zu der das Objekt gehört. & 1 \tabularnewline \hline

\end{tabular*}
\label{tab:schema_recordings}
\end{table}

% This file was generated Wed Nov 03 17:02:41 +0100 2010
%!TEX root = ../c:/Dokumente und Einstellungen/rf/github/diplom.rails/lib/tasks/../../doc/tex/doc.tex

\begin{table}[h]
  \begin{center}
    \begin{tabular}{|l|p{.09\textwidth}|p{.4\textwidth}|l|} \hline
      \textbf{Name} & \textbf{Typ}  & \textbf{Beschreibung}  & \textbf{Beispiel} \tabularnewline \hline

      name & String & Name der Region & "`"' \tabularnewline \hline

      country\_id & Integer & Verweis auf das Land, zum dem die Region gehört & 1 \tabularnewline \hline

    \end{tabular}
    \caption{Struktur der "`regions"'-Datenbanktabelle}
    \label{tab:schema_regions}
  \end{center}
\end{table}

% This file was generated Mon Nov 01 23:38:23 +0100 2010
%!TEX root = ..//Users/rene/Sites/github/diplom.rails/lib/tasks/../../doc/tex/doc.tex

\begin{table}[h]
  \begin{center}
    \caption{Struktur der "`regions\_stations"'-Datenbanktabelle}
    \begin{tabular}{|l|l|l|l|} \hline
      \textbf{Name} & \textbf{Typ}  & \textbf{Beschreibung}  & \textbf{Beispiel} \tabularnewline \hline
      region\_id & Integer &  & "`"' \tabularnewline \hline
      station\_id & Integer &  & "`"' \tabularnewline \hline
    \end{tabular}
    \label{tab:schema_regions_stations}
  \end{center}
\end{table}

% This file was generated Sun Oct 31 19:39:54 +0100 2010
%!TEX root = ..//Users/rene/Sites/github/diplom.rails/lib/tasks/../../doc/tex/doc.tex

\begin{table}[h]
\caption{Aufbau der "`reviews"'-Datenbanktabelle}
\begin{tabular*}{\textwidth}{|l|l|p{7cm}|l|} \hline
  \textbf{Name} & \textbf{Typ}  & \textbf{Beschreibung}  & \textbf{Beispiel} \tabularnewline \hline
  author\_user\_id & Integer &  &  \tabularnewline \hline
  author\_company\_id & Integer &  &  \tabularnewline \hline
  approved\_by\_id & Integer &  &  \tabularnewline \hline
  company\_id & Integer & Verweis auf die Firma, zu der das Objekt gehört. & 1 \tabularnewline \hline
  text & Text &  &  \tabularnewline \hline
\end{tabular*}
\label{tab:schema_reviews}
\end{table}

% This file was generated Sun Oct 31 11:50:40 +0100 2010
%!TEX root = ../c:/Dokumente und Einstellungen/rf/github/diplom.rails/lib/tasks/../../doc/tex/doc.tex

\begin{table}[h]
\caption{Felder der "`search\_recordings"'-Tabelle}
\begin{tabular*}{}{|l|l|p{7cm}|l|} \hline
  \textbf{Name} & \textbf{Typ}  & \textbf{Beschreibung}  & \textbf{Beispiel} \tabularnewline \hline

  user\_id & Integer & Verweis auf den Benutzer, dem das Objekt gehört. & 1 \tabularnewline \hline

  query & String &  &  \tabularnewline \hline

  results & Integer &  &  \tabularnewline \hline

  parent\_id & Integer &  &  \tabularnewline \hline

  result\_type & String &  &  \tabularnewline \hline

  result\_id & Integer &  &  \tabularnewline \hline

\end{tabular*}
\label{tab:schema_search_recordings}
\end{table}

% This file was generated Wed Nov 03 17:02:41 +0100 2010
%!TEX root = ../c:/Dokumente und Einstellungen/rf/github/diplom.rails/lib/tasks/../../doc/tex/doc.tex

\begin{table}[h]
  \begin{center}
    \begin{tabular}{|l|p{.09\textwidth}|p{.4\textwidth}|l|} \hline
      \textbf{Name} & \textbf{Typ}  & \textbf{Beschreibung}  & \textbf{Beispiel} \tabularnewline \hline

      item\_type & String &  & "`"' \tabularnewline \hline

      item\_id & Integer &  & "`"' \tabularnewline \hline

      text & Text &  & "`"' \tabularnewline \hline

    \end{tabular}
    \caption{Struktur der "`simple\_searches"'-Datenbanktabelle}
    \label{tab:schema_simple_searches}
  \end{center}
\end{table}

% This file was generated Mon Nov 01 23:38:23 +0100 2010
%!TEX root = ..//Users/rene/Sites/github/diplom.rails/lib/tasks/../../doc/tex/doc.tex

\begin{table}[h]
  \begin{center}
    \caption{Struktur der "`site\_infos"'-Datenbanktabelle}
    \begin{tabular}{|l|l|l|l|} \hline
      \textbf{Name} & \textbf{Typ}  & \textbf{Beschreibung}  & \textbf{Beispiel} \tabularnewline \hline
      contractor & String &  & "`"' \tabularnewline \hline
      name & String &  & "`"' \tabularnewline \hline
      address & String &  & "`"' \tabularnewline \hline
      address2 & String &  & "`"' \tabularnewline \hline
      zip & String &  & "`"' \tabularnewline \hline
      city & String &  & "`"' \tabularnewline \hline
      country & String &  & "`"' \tabularnewline \hline
      side\_track\_available & Boolean &  & "`"' \tabularnewline \hline
      track\_number & String &  & "`"' \tabularnewline \hline
    \end{tabular}
    \label{tab:schema_site_infos}
  \end{center}
\end{table}

% This file was generated Wed Nov 03 17:02:41 +0100 2010
%!TEX root = ../c:/Dokumente und Einstellungen/rf/github/diplom.rails/lib/tasks/../../doc/tex/doc.tex

\begin{table}[h]
  \begin{center}
    \begin{tabular}{|l|p{.09\textwidth}|p{.4\textwidth}|l|} \hline
      \textbf{Name} & \textbf{Typ}  & \textbf{Beschreibung}  & \textbf{Beispiel} \tabularnewline \hline

      name & String &  & "`"' \tabularnewline \hline

      country\_id & Integer &  & "`"' \tabularnewline \hline

      address & String &  & "`"' \tabularnewline \hline

      address2 & String &  & "`"' \tabularnewline \hline

      zip & String &  & "`"' \tabularnewline \hline

      city & String &  & "`"' \tabularnewline \hline

    \end{tabular}
    \caption{Struktur der "`stations"'-Datenbanktabelle}
    \label{tab:schema_stations}
  \end{center}
\end{table}

% This file was generated Wed Nov 03 17:02:41 +0100 2010
%!TEX root = ../c:/Dokumente und Einstellungen/rf/github/diplom.rails/lib/tasks/../../doc/tex/doc.tex

\begin{table}[h]
  \begin{center}
    \begin{tabular}{|l|p{.09\textwidth}|p{.4\textwidth}|l|} \hline
      \textbf{Name} & \textbf{Typ}  & \textbf{Beschreibung}  & \textbf{Beispiel} \tabularnewline \hline

      name & String & Name der Benutzerrolle & "`company\_admin"' \tabularnewline \hline

    \end{tabular}
    \caption{Struktur der "`user\_roles"'-Datenbanktabelle}
    \label{tab:schema_user_roles}
  \end{center}
\end{table}

% This file was generated Sun Oct 31 19:39:54 +0100 2010
%!TEX root = ..//Users/rene/Sites/github/diplom.rails/lib/tasks/../../doc/tex/doc.tex

\begin{table}[h]
\caption{Aufbau der "`user\_roles\_users"'-Datenbanktabelle}
\begin{tabular*}{\textwidth}{|l|l|p{7cm}|l|} \hline
  \textbf{Name} & \textbf{Typ}  & \textbf{Beschreibung}  & \textbf{Beispiel} \tabularnewline \hline
  user\_id & Integer & Verweis auf den Benutzer, dem das Objekt gehört. & 1 \tabularnewline \hline
  user\_role\_id & Integer &  &  \tabularnewline \hline
\end{tabular*}
\label{tab:schema_user_roles_users}
\end{table}

% This file was generated Thu Nov 04 14:11:13 +0100 2010
%!TEX root = ../c:/Dokumente und Einstellungen/rf/github/diplom.rails/lib/tasks/../../doc/tex/doc.tex

\begin{table} %[h]
  \begin{center}
    \begin{tabular}{|p{.3\textwidth}|p{.09\textwidth}|p{.26\textwidth}|p{.24\textwidth}|} \hline
      \textbf{Name} & \textbf{Typ}  & \textbf{Beschreibung}  & \textbf{Beispiel} \tabularnewline \hline

      login & String & Benutzername & "`max.mustermann"' \tabularnewline \hline

      email & String & E-Mail & "`mm@example.org"' \tabularnewline \hline

      crypted\_password & String & Verschlüsseltes Passwort & "`55024db979..."' \tabularnewline \hline

      password\_salt & String & Geheimer Passwortschlüssel & "`55024db979..."' \tabularnewline \hline

      persistence\_token & String &  & "`"' \tabularnewline \hline

      single\_access\_token & String &  & "`"' \tabularnewline \hline

      perishable\_token & String &  & "`"' \tabularnewline \hline

      login\_count & Integer & Anzahl Logins & 120 \tabularnewline \hline

      failed\_login\_count & Integer & Fehlgeschlagene Loginversuche & 13 \tabularnewline \hline

      current\_login\_ip & String & Aktuelle IP & "`192.188.142.11"' \tabularnewline \hline

      last\_login\_ip & String & Letzte IP & "`192.188.142.11"' \tabularnewline \hline

      company\_id & Integer & Verweis auf die Firma, zu der der Benutzer gehört. & 1 \tabularnewline \hline

      person\_id & Integer & Verweis auf die Person, zu der der Benutzer gehört. & 1 \tabularnewline \hline

      api\_key & String & Alphanumberischer Schlüssel zur Ansteuerung der XML/JSON-API & "`55024db979..."' \tabularnewline \hline

      posting\_type & String &  &  \tabularnewline \hline

    \end{tabular}
    \caption{Struktur der "`users"'-Datenbanktabelle}
    \label{tab:schema_users}
  \end{center}
\end{table}


%%%begin_includes%%%
\include{classes/README_FOR_APP}
\include{classes/ActiveRecord-Base}
\include{classes/Admin-AppConfigsController}
\include{classes/Admin-BaseController}
\include{classes/Admin-BaseHelper}
\include{classes/Admin-StationsController}
\include{classes/Admin-UserRolesController}
\include{classes/AppConfig}
\include{classes/ApplicationController}
\include{classes/ApplicationHelper}
\include{classes/CompaniesController}
\include{classes/CompaniesHelper}
\include{classes/Company}
\include{classes/Country}
\include{classes/ErrorMessages}
\include{classes/Freight}
\include{classes/FreightsHelper}
\include{classes/GeneralObserver}
\include{classes/InheritedResources-Base}
\include{classes/LoadingSpace}
\include{classes/LoadingSpacesHelper}
\include{classes/LocalizedInfo}
\include{classes/Matching-Compare-Base}
\include{classes/Matching-Compare-Fixnum}
\include{classes/Matching-Compare-FreightToLoadingSpace}
\include{classes/Matching-Compare-Hash}
\include{classes/Matching-Compare-SiteInfo}
\include{classes/Matching-Compare-String}
\include{classes/Matching-Compare-Time}
\include{classes/Matching-Compare}
\include{classes/Matching}
\include{classes/MatchingRecording}
\include{classes/Object}
\include{classes/PeopleController}
\include{classes/Person}
\include{classes/Posting-InstanceMethods}
\include{classes/Posting}
\include{classes/PostingsController}
\include{classes/Recording}
\include{classes/ReviewsController}
\include{classes/RootController}
\include{classes/Search}
\include{classes/SearchController}
\include{classes/SiteInfo}
\include{classes/Station}
\include{classes/User}
\include{classes/UserRole}
\include{classes/UsersController}
\include{classes/UserSession}
\include{classes/UserSessionsController}
%%%end_includes%%%

\end{document}