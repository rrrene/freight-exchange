
%Achtung: Wenn eps Grafiken im Dokument enthalten sind von pdftex auf tex und ghostsript umstellen (Im Men� Setzen von TeXShop)
%In dieser Datei sind die Graphiken auskommentiert.

%Hier steht die Pr�ambel
\documentclass[a4paper,parskip=half*,12pt]{scrartcl} 
%definiert Papiergr��e und Dokumentklasse (KOMA Script)

%Die folgenden Zeilen sind der �bersichtlichkeit halber so gestaltet
%Alternativ lassen sich auch mehrere Pakte zusammenfassen:
%\usepackage{jurabib,hyperref,graphicx,setspace}

\usepackage[ngerman]{babel}
\usepackage[utf8]{inputenc}
%Erlaubt die direkte Eingabe von z.B. Umlauten, zur Aktivierung das %-Zeichen entfernen

\usepackage[paper=a4paper,left=40mm,right=10mm,top=25mm,bottom=10mm]{geometry} 

\usepackage[T1]{fontenc}

\usepackage{hyperref}
%Ein paar weitere Pakete f�r dies und das ;-)

\usepackage{graphicx} 
%Wenn Bilder eingef�gt werden sollen muss da drinstehen

\usepackage{setspace}
%Wird f�r den Zeilenabstand ben�tigt

%einzelne, herrenlose Zeilen (Hurenkinder und Schusterjungen) vermeiden
\clubpenalty = 10000
\widowpenalty = 10000
\displaywidowpenalty = 10000

%wie weit sollen paragraphen einger�ckt werden
\setlength{\parindent}{0em}

%Auf die Einr�ckung der Fu�noten wird verzichtet
\deffootnote{1em}{1em}{%
\textsuperscript{\thefootnotemark\ }}

%Bis zu welcher Gliederungsebene soll nummeriert werden 
\setcounter{secnumdepth}{3}

\setcounter{tocdepth}{3}


%Kopf- und Fu�zeile
\usepackage{fancyhdr}
\pagestyle{fancy} 
\fancyhf{} 
\fancyhead[L]{\nouppercase{\leftmark}}%Kopfzeile links, zentriert: [C] 
\fancyhead[R]{\thepage} %Kopfzeile rechts, links:

\usepackage{color}
\newcommand{\bad}[1]{\textcolor{red}{#1}}

% Ak�rzungen und Acronyme kennzeichnen
\newcommand{\aav}[2]{#1}
\newcommand{\quelle}[1]{\textsc{#1}}
%\aav{vgl.}{vergleiche}
%\aav{S.}{Seite}
%\aav{et al.}{et alii}
%\aav{f.}{folgende}
%\aav{ff.}{fortfolgende}
%\aa%v{o.S.}{ohne Seitenangabe}
%\aa%v{o.V.}{ohne Verfasserangabe}
%\aav{}{}
%

%Jetzt beginnt das eigentliche Dokument
\begin{document}

%ab jetzt sollen die Seitenzahlen beginnen, mit 1 f�r die erste Seite, usw.
\setcounter{page}{1}
\pagenumbering{arabic}

%ab jetzt anderthalbfacher Zeilenabstand
\onehalfspacing

%%%begin_includes%%%
\include{classes/README_FOR_APP}
\include{classes/ActiveRecord-Base}
\include{classes/ActiveRecord}
\include{classes/AppConfig}
\include{classes/ApplicationController}
\include{classes/ApplicationHelper}
\include{classes/CompaniesController}
\include{classes/CompaniesHelper}
\include{classes/Company}
\include{classes/Country}
\include{classes/Freight}
\include{classes/FreightsController}
\include{classes/FreightsHelper}
\include{classes/GeneralObserver}
\include{classes/LoadingSpace}
\include{classes/LoadingSpacesController}
\include{classes/LoadingSpacesHelper}
\include{classes/LocalizedInfo}
\include{classes/Matching-Compare-Base}
\include{classes/Matching-Compare-Fixnum}
\include{classes/Matching-Compare-FreightToLoadingSpace}
\include{classes/Matching-Compare-Hash}
\include{classes/Matching-Compare-SiteInfo}
\include{classes/Matching-Compare-String}
\include{classes/Matching-Compare-Time}
\include{classes/Matching-Compare}
\include{classes/Matching}
\include{classes/Object}
\include{classes/PeopleController}
\include{classes/PeopleHelper}
\include{classes/Person}
\include{classes/Posting}
\include{classes/PostingsController}
\include{classes/PostingsHelper}
\include{classes/Recording}
\include{classes/Region}
\include{classes/RootController}
\include{classes/RootHelper}
\include{classes/Search}
\include{classes/SearchController}
\include{classes/SearchHelper}
\include{classes/SiteInfo}
\include{classes/Station}
\include{classes/StationsController}
\include{classes/StationsHelper}
\include{classes/User}
\include{classes/UserRole}
\include{classes/UsersController}
\include{classes/UserSession}
\include{classes/UserSessionsController}
\include{classes/UserSessionsHelper}
\include{classes/UsersHelper}
%%%end_includes%%%

\end{document}