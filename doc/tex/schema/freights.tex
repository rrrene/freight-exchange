% This file was generated Wed Nov 03 17:02:41 +0100 2010
%!TEX root = ../c:/Dokumente und Einstellungen/rf/github/diplom.rails/lib/tasks/../../doc/tex/doc.tex

\begin{table}[h]
  \begin{center}
    \begin{tabular}{|p{.3\textwidth}|p{.09\textwidth}|p{.35\textwidth}|p{.15\textwidth}|} \hline
      \textbf{Name} & \textbf{Typ}  & \textbf{Beschreibung}  & \textbf{Beispiel} \tabularnewline \hline

      user\_id & Integer & Verweis auf den Benutzer, dem das Objekt gehört. & 1 \tabularnewline \hline

      company\_id & Integer & Verweis auf die Firma, zu der das Objekt gehört. & 1 \tabularnewline \hline

      origin\_site\_info\_id & Integer & Verweis auf den Startort & 34 \tabularnewline \hline

      destination\_site\_info\_id & Integer & Verweis auf den Zielort & 35 \tabularnewline \hline

      weight & Integer & Gewicht der Fracht (in t) & 50 \tabularnewline \hline

      loading\_meter & Integer & Lademeter & 30 \tabularnewline \hline

      hazmat & Boolean & Ist das Gut ein Gefahrgut? & true \tabularnewline \hline

      transport\_type & String & Art der Wagen & single\_wagon \tabularnewline \hline

      wagons\_provided\_by & String & Wer stellt die Wagen bereit? & railway \tabularnewline \hline

      desired\_proposal\_type & String & Welche Art von Angebot wird gewünscht? & package\_price \tabularnewline \hline

      contact\_person\_id & Integer & Verweis auf die Person, die Ansprechpartner sein soll. & 1 \tabularnewline \hline

    \end{tabular}
    \caption{Struktur der "`freights"'-Datenbanktabelle}
    \label{tab:schema_freights}
  \end{center}
\end{table}
