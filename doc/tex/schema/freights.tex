% This file was generated Sun Oct 31 19:39:54 +0100 2010
%!TEX root = ..//Users/rene/Sites/github/diplom.rails/lib/tasks/../../doc/tex/doc.tex

\begin{table}[h]
\caption{Aufbau der "`freights"'-Datenbanktabelle}
\begin{tabular*}{\textwidth}{|l|l|p{7cm}|l|} \hline
  \textbf{Name} & \textbf{Typ}  & \textbf{Beschreibung}  & \textbf{Beispiel} \tabularnewline \hline
  user\_id & Integer & Verweis auf den Benutzer, dem das Objekt gehört. & 1 \tabularnewline \hline
  company\_id & Integer & Verweis auf die Firma, zu der das Objekt gehört. & 1 \tabularnewline \hline
  origin\_site\_info\_id & Integer &  &  \tabularnewline \hline
  destination\_site\_info\_id & Integer &  &  \tabularnewline \hline
  weight & Integer & Gewicht der Fracht (in t) & 50 \tabularnewline \hline
  loading\_meter & Integer & Lademeter & 30 \tabularnewline \hline
  hazmat & Boolean & Ist das Gut ein Gefahrgut? & true \tabularnewline \hline
  transport\_type & String &  &  \tabularnewline \hline
  wagons\_provided\_by & String & Wer stellt die Wagen bereit? &  \tabularnewline \hline
  desired\_proposal\_type & String & Welche Art von Angebot wird gewünscht? &  \tabularnewline \hline
\end{tabular*}
\label{tab:schema_freights}
\end{table}
